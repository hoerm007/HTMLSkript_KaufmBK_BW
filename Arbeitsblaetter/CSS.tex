% !TeX root = ../Skript_HTML.tex
\cohead{\Large\textbf{CSS}}
\section{Cascading Style Sheets}
Es gibt zwar noch immer HTML-Tags, die zur Formatierung der Webseite dienen, wie z.B. Ändern der Schriftfarbe, jedoch sind so gut wie alle diese Tags als veraltet (deprecated) markiert. Veraltete Tags werden zwar noch unterstützt, sollten aber nicht mehr verwendet werden. In ferner Zukunft werden Browser diese Tags nicht mehr unterstützen wodurch es dann zu Darstellungsfehlern im Browser kommen wird.

Der grundlegende Style (Formatierung abgesehen von physischen Textauszeichnungen) wird über das sogenannte CSS festgelegt. CSS steht für Cascading Style Sheets, was zusätzliche Dateien impliziert. Dies ist jedoch nicht immer notwendig. CSS lässt sich auf 3 möglichen Wegen verwenden:
\begin{enumerate}
    \item Inline, indem man ein style-Attribut innerhalb eines HTML-Tags verwendet, falls man tatsächlich nur die Formatierung dieses einen Tags ändern will, z.B. wird \lstinline|<strong style="color:red;">Beispieltext</strong>| hervorgehoben und Rot dargestellt.
    \item Intern innerhalb einer Datei, indem man ein \lstinline|<style>|-Tag in den \lstinline|<head>|-Bereich der HTML-Datei einfügt, falls man die Formatierung innerhalb dieser einen Datei ändern will. Diese Einträge sind mit denen, die wir in einer CSS-Datei anlegen werden identisch, nur, dass sie in die HTML-Datei geschrieben werden.
    \item Extern, indem man über ein  \lstinline|<link>|-Tag im \lstinline|<head>|-Bereich der HTML-Datei eine externe CSS-Datei einbindet. Diese Datei kann man in verschiedenen HTML-Dateien einbinden und so die Formatierung für mehrere Webseiten zentral verwalten.
\end{enumerate}

Wir werden nur die externe Variante verwenden. Die anderen Varianten sind von der Verwendung her gleich nur der Ort, an dem man die Formatierungsanweisungen angibt, ändert sich.

\begin{Exercise}[title=, label=CSS1]
    \begin{enumerate}
        \item Erzeuge mit dem Editor eine leere Datei und speichere diese unter \textit{style.css} im selben Ordner wie \textit{schueler.html}. Auf die Dateiendung achten! Nicht als \textit{style.css.txt} speichern.
        \item Binde die CSS-Datei in die Schülerseite im  \lstinline|<head>|-Bereich von \textit{schueler.html} ein:
        \lstinline|<link href="style.css" rel="stylesheet" type="text/css">|
    \end{enumerate}
\end{Exercise}
In der CSS-Datei kann man nun für verschiedene Tags die Formatierung ändern. Die Syntax einer CSS-Datei dazu sieht wie folgt aus:

\lstinline|Tagname (mehrere kommaseparierte Tags zulässig){Eigenschaft:neuer Wert;}|

Will man z.B. die Farbe der größten Überschriften in Blau ändern, so fügt man folgenden Eintrag hinzu: \lstinline|h1 {color:blue;}|

Man kann in den geschweiften Klammern beliebig viele Eigenschaften ändern. Die Namen und zulässigen Werte können z.B. bei w3schools gefunden werden.

\begin{Exercise}[title=, label=CSS2]
    Nimm (über die CSS-Datei) folgende Änderungen für die bisherige Seite vor:
    \begin{enumerate}
        \item Die Hintergrundfarbe soll auf gelb gesetzt werden (informiere dich in diesem Zusammenhang über die Farbangabe mit Hilfe des RGB-Codes).
        \item Alle Überschriften sollen immer Rot und unterstrichen dargestellt werden.
        \item Die Überschriften ersten Grades sollen immer zentriert ausgegeben werden.
    \end{enumerate}
    Validiere dann deine Datei \textit{schueler.html}.
\end{Exercise}

