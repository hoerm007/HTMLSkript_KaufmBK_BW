% !TeX root = ../Skript_HTML.tex
\cohead{\Large\textbf{Lösungen}}
\section{Lösungen}
Anbei Beispiele für die HTML bzw. CSS-Dateien nach jedem Kapitel. Insbesondere die CSS-Datei kann auch anders aufgebaut werden.
\begin{enumerate}
    \item Begriffe und Grundgerüst

    \textit{schuler.html}:
    \begin{lstlisting}
<html>
<head>
<title>Schülerseite</title>
</head>
<body>
Willkommen auf meiner Webseite!
<br>
<br>
<p>Ich bin Schüler des WG West in Stuttgart.</p>
</body>
</html>
    \end{lstlisting}
    \item Umlaute und Sonderzeichen

    \textit{schuler.html}:
    \begin{lstlisting}
<html>
<head>
<title>Schülerseite</title>
<meta charset="utf-8">
</head>
<body>
Willkommen auf meiner Webseite!
<br>
<br>
<p>Ich bin Schüler des WG West in Stuttgart.</p>
</body>
</html>
    \end{lstlisting}
    \item Validieren

    \textit{schuler.html}:
    \begin{lstlisting}
<!DOCTYPE html>
<html lang="de">
<head>
<title>Schülerseite</title>
<meta charset="utf-8">
</head>
<body>
Willkommen auf meiner Webseite!
<br>
<br>
<p>Ich bin Schüler des WG West in Stuttgart.</p>
</body>
</html>
    \end{lstlisting}
    \item Textauszeichnungen

    \textit{schuler.html}:
    \begin{lstlisting}
<!DOCTYPE html>
<html lang="de">
<head>
<title>Sch&uuml;lerseite</title>
<meta charset="utf-8">
</head>
<body>
<h1>Haralds Webseite</h1>
<h2>Willkommen auf meiner Webseite!</h2>
<br>
<hr>
<br>
<p>Ich bin Sch&uuml;ler des <strong>WG West in Stuttgart</strong>.</p>
</body>
</html>
    \end{lstlisting}
    \item Cascading Style Sheets

    \textit{schuler.html}:
    \begin{lstlisting}
<!DOCTYPE html>
<html lang="de">
<head>
<link href="style.css" rel="stylesheet" type="text/css">
<title>Sch&uuml;lerseite</title>
<meta charset="utf-8">
</head>
<body>
<h1>Haralds Webseite</h1>
<h2>Willkommen auf meiner Webseite!</h2>
<br>
<hr>
<br>
<p>Ich bin Sch&uuml;ler des <strong>WG West in Stuttgart</strong>.</p>
</body>
</html>
    \end{lstlisting}
    \textit{style.css}
    \begin{lstlisting}
body {background-color: yellow;}
h1 {text-align: center;}
h1, h2 {color: red}
    \end{lstlisting}
    \item Listen

    \textit{schuler.html}:
    \begin{lstlisting}
<!DOCTYPE html>
<html lang="de">
<head>
<link href="style.css" rel="stylesheet" type="text/css">
<title>Sch&uuml;lerseite</title>
<meta charset="utf-8">
</head>
<body>
<h1>Haralds Webseite</h1>
<h2>Willkommen auf meiner Webseite!</h2>
<br>
<hr>
<br>
<p>Ich bin Sch&uuml;ler des <strong>WG West in Stuttgart</strong>.</p>
<br>
Was ich <strong>gut</strong> finde:
<ul>
<li>Informatik-Unterricht</li>
<li>HTML</li>
<li>CSS</li>
</ul>
Was ich <em>nicht</em> gut finde:
<ol>
<li>Listen</li>
<li>Ironie</li>
<li>Wiederholungen</li>
<li>Listen</li>
</ol>
</body>
</html>
    \end{lstlisting}
    \textit{style.css}
    \begin{lstlisting}
body {background-color: yellow;}
h1 {text-align: center;}
h1, h2 {color: red;}
ul {list-style-type: square;}
ol {list-style-type: upper-roman; color: green;}
    \end{lstlisting}
    \item Bilder

    \textit{schuler.html}:
    \begin{lstlisting}
<!DOCTYPE html>
<html lang="de">
<head>
<link href="style.css" rel="stylesheet" type="text/css">
<title>Sch&uuml;lerseite</title>
<meta charset="utf-8">
</head>
<body>
<h1>Haralds Webseite</h1>
<img src="Bilder/Bild.png" alt="Ein von canva.com erzeugtes KI-Bild" height="200">
<h2>Willkommen auf meiner Webseite!</h2>
<br>
<hr>
<br>
<p>Ich bin Sch&uuml;ler des <strong>WG West in Stuttgart</strong>.</p>
<br>
Was ich <strong>gut</strong> finde:
<ul>
<li>Informatik-Unterricht</li>
<li>HTML</li>
<li>CSS</li>
</ul>
Was ich <em>nicht</em> gut finde:
<ol>
<li>Listen</li>
<li>Ironie</li>
<li>Wiederholungen</li>
<li>Listen</li>
</ol>
</body>
</html>
    \end{lstlisting}
    \item Abschnitte definieren und formatieren

    \textit{schuler.html}:
    \begin{lstlisting}
<!DOCTYPE html>
<html lang="de">
<head>
<link href="style.css" rel="stylesheet" type="text/css">
<title>Sch&uuml;lerseite</title>
<meta charset="utf-8">
</head>
<body>
<div class="meineErsteKlasse">
<h1>Haralds Webseite</h1>
<img src="Bilder/Bild.png" alt="Ein von canva.com erzeugtes KI-Bild" height="200">
<h2>Willkommen auf meiner Webseite!</h2>
<br>
<hr>
<br>
<p>Ich bin Sch&uuml;ler des <strong>WG West in Stuttgart</strong>.
</div>
<br>
Was ich <strong>gut</strong> finde:
<ul>
<li>Informatik-Unterricht</li>
<li>HTML</li>
<li>CSS</li>
</ul>
Was ich <em>nicht</em> gut finde:
<ol>
<li>Listen</li>
<li>Ironie</li>
<li>Wiederholungen</li>
<li>Listen</li>
</ol>
</body>
</html>
    \end{lstlisting}
    \textit{style.css}
    \begin{lstlisting}
    body {background-color: yellow;}
h1 {text-align: center;}
h1, h2 {color: red;}
ul {list-style-type: square;}
ol {list-style-type: upper-roman; color: green;}

.meineErsteKlasse {
    text-align: center;
}
    \end{lstlisting}
    \item Verlinkungen

    \textit{schuler.html}:
    \begin{lstlisting}
<!DOCTYPE html>
<html lang="de">
<head>
<link href="style.css" rel="stylesheet" type="text/css">
<title>Sch&uuml;lerseite</title>
<meta charset="utf-8">
</head>
<body>
<div class="meineErsteKlasse">
<h1>Haralds Webseite</h1>
<img src="Bilder/Bild.png" alt="Ein von canva.com erzeugtes KI-Bild" height="200">
<h2>Willkommen auf meiner Webseite!</h2>
<br>
<hr>
<br>
<p>Ich bin <a href="#SchuelerBild">Sch&uuml;ler</a> des <strong><a href="https://www.wg-west.de/">WG West in Stuttgart</a></strong>.
</div>
<br>
Was ich <strong>gut</strong> finde:
<ul>
<li>Informatik-Unterricht</li>
<li>HTML</li>
<li>CSS</li>
</ul>
Was ich <em>nicht</em> gut finde:
<ol>
<li>Listen</li>
<li>Ironie</li>
<li>Wiederholungen</li>
<li>Listen</li>
</ol>
Das ist meine Klasse:
<br>
<img id="SchuelerBild" src="Bilder/Klasse.png" alt="Ein von canva.com erzeugtes KI-Bild meiner Klasse" height="400"><br>
Das ist mein <a href="stundenplan.html">Stundenplan</a>.
</body>
</html>
    \end{lstlisting}
    \textit{stundenplan.html}
    \begin{lstlisting}
<!DOCTYPE html>
<html lang="de">
<head>
<link href="style.css" rel="stylesheet" type="text/css">
<title>Unterricht</title>
<meta charset="utf-8">
</head>
<body>
So sieht mein Stundenplan aus:
</body>
</html>
    \end{lstlisting}
    \item Tabellen

    \textit{stundenplan.html}:
    \begin{lstlisting}
<!DOCTYPE html>
<html lang="de">
<head>
<link href="style.css" rel="stylesheet" type="text/css">
<title>Unterricht</title>
<meta charset="utf-8">
</head>
<body>
So sieht mein Stundenplan aus:<br>
<table>
<tr>
<th></th>
<th>Mo</th>
<th>Di</th>
<th>Mi</th>
<th>Do</th>
<th>Fr</th>
</tr>
<tr>
<td>1./2. Stunde</td>
<td>Mathe</td>
<td>BWL</td>
<td>BWL</td>
<td>Deutsch</td>
<td>Üfa</td>
</tr>
<tr>
<td>3./4. Stunde</td>
<td>Informatik</td>
<td>Wirtschaft</td>
<td>Englisch</td>
<td>Reli</td>
<td>Üfa</td>
</tr>
<tr>
<td>5./6. Stunde</td>
<td>Deutsch</td>
<td>GGK</td>
<td>SK</td>
<td></td>
<td>Üfa</td>
</tr>
<tr>
<td colspan="6">Hier gehts <a href="schueler.html">zur&uuml;ck</a></td>
</tr>
</table>
</body>
</html>
    \end{lstlisting}
    \textit{style.css}
    \begin{lstlisting}
body {background-color: yellow;}
h1 {text-align: center;}
h1, h2 {color: red;}
ul {list-style-type: square;}
ol {list-style-type: upper-roman; color: green;}
table, th, td {border: 1px solid black;}
th {font-style: italic;}
td, th {text-align: center; color:blue;}
table {width:800px;}

.meineErsteKlasse {text-align: center;}
    \end{lstlisting}
    \item Hover
    
    \textit{stundenplan.html}:
    \begin{lstlisting}
<!DOCTYPE html>
<html lang="de">
<head>
<link href="style.css" rel="stylesheet" type="text/css">
<title>Unterricht</title>
<meta charset="utf-8">
</head>
<body>
So sieht mein Stundenplan aus:<br>
<table>
<tr>
<th></th>
<th>Mo</th>
<th>Di</th>
<th>Mi</th>
<th>Do</th>
<th>Fr</th>
</tr>
<tr>
<td>1./2. Stunde</td>
<td>Mathe</td>
<td>BWL</td>
<td>BWL</td>
<td>Deutsch</td>
<td>Üfa</td>
</tr>
<tr>
<td>3./4. Stunde</td>
<td class="roterHintergrund">Informatik</td>
<td>Wirtschaft</td>
<td>Englisch</td>
<td>Reli</td>
<td>Üfa</td>
</tr>
<tr>
<td>5./6. Stunde</td>
<td>Deutsch</td>
<td>GGK</td>
<td>SK</td>
<td></td>
<td>Üfa</td>
</tr>
<tr>
<td colspan="6">Hier gehts <a href="schueler.html">zur&uuml;ck</a></td>
</tr>
</table>
</body>
</html>
    \end{lstlisting}
    \textit{style.css}
    \begin{lstlisting}
body {background-color: yellow;}
h1 {text-align: center;}
h1, h2 {color: red;text-decoration-line: underline;}
ul {list-style-type: square;}
ol {list-style-type: upper-roman; color: green;}
table, th, td {border: 1px solid black;}
th {font-style: italic;}
td, th {text-align: center; color:blue;}
table {width:800px;}

.meineErsteKlasse {text-align: center;}
th:hover {background-color: red;}
td.roterHintergrund:hover {background-color: green; font-style: italic; color:yellow;}
    \end{lstlisting}
\end{enumerate}
