% !TeX root = ../Skript_HTML.tex
\cohead{\Large\textbf{Hover}}
\section{Hover}
Der Name kommt wie üblich aus dem Englischen von to hover above something, über etwas schweben. Mit Hilfe dieses Tags kann man Teile der Webseite ändern, falls der Mauszeiger über diesem Teil "schwebt". 

\begin{Exercise}[title=Erstelle deinen Stundenplan in \textit{stundenplan.html}, label=Hover1]
	\begin{enumerate}
		\item Informiere dich über die Verwendung der Pseudoklasse Hover (Hinweis: Man muss Eintragungen in der CSS-Datei vornehmen, bevor man es in der HTML-Datei verwenden kann).
		\item Ändere deine Webseite so, dass der Hintergrund der Überschriften der Stundenplantabelle rot wird, wenn man mit dem Mauszeiger über die Zellen fährt.
		\item Ändere deine Webseite so, dass nur die Zelle mit dem Inhalt "Informatik" in der Stundenplantabelle grün wird, der Text kursiv und die Farbe des Texts gelb, wenn man mit dem Mauszeiger über die Zelle fährt.
	\end{enumerate}
\end{Exercise}

