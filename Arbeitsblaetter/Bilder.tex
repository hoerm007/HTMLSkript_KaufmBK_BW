% !TeX root = ../Skript_HTML.tex
\cohead{\Large\textbf{Bilder}}
\section{Bilder}
Eine Webseite lebt von Bildern. Um die Webseite optisch ansprechender zu machen, wird ein Bild eingefügt (Das Bild wurde mittels \href{https://www.canva.com/}{canva.com} KI-erzeugt):
\begin{minipage}[t]{\textwidth}
    \includegraphics[width=\linewidth]{\pics/BilderEinfuegen.png}
\end{minipage}
\begin{Exercise}[title=, label=Bilder]
    \begin{enumerate}
        \item Lade dir Zwei Bilder aus dem Internet herunter und speichere diese in einem Ordner Bilder, der im gleichen Verzeichnis wie \textit{schueler.html} liegt. Kurze Dateinamen machen das Einbinden leichter.
        \item Binde in deiner \textit{schueler.html} eines der beiden Bilder ein. Recherchiere dazu den Tag für das Einfügen von Bildern. Erkundige dich, welche Attribute für die Höhen- und Breitenangabe von Bildern notwendig sind und was der Alternativtext bei Bildern bedeutet.
    \end{enumerate}
    Informiere dich über die wesentlichen Bildformate (Rasterbilder wie jpg, gif, png oder Vektorgrafiken wie svg) und deren Unterschiede. Wichtige Begriffe sind hier unter anderem Dateigröße, Komprimierung, Transparenz, Farbraum.
\end{Exercise}


