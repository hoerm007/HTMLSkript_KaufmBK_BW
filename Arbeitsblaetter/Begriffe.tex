% !TeX root = ../Skript_HTML.tex
\cohead{\Large\textbf{Begriffe und Grundgerüst}}
\section{Begriffe und Grundgerüst}
Das Internet ist aus unserer heutigen Welt nicht mehr wegzudenken. Im WWW findet man Informationen jeder Art, man kauft ein, trifft sich und kommuniziert rund um den ganzen Globus.  Das WWW unter Eingabe der URL zu nutzen, ist heute selbstverständlich geworden. Eine Webseite jedoch mit HTML zu erstellen, welche von einem Browser angezeigt werden kann, beherrschen nur wenige. Um die Grundlagen von HTML Schritt für Schritt zu erlernen, werden wir eine eigene Webseite erstellen. Wir beschränken uns auf die Grundlagen, d.h. wir werden einen einfachen Texteditor zum Erstellen verwenden anstatt einer Entwicklungsumgebung. Zum Anzeigen der Webseite werden wir einen normalen Browser wie Chrome, Firefox oder Edge verwenden.

Eine Webseite ist letztendlich eine Datei. Standardmäßig ist diese Datei in der Auszeichnungssprache HTML erstellt. Wir werden also selbst eine solche Datei erstellen. Greift man mit Hilfe eines Browsers über das Internet auf eine Seite zu, so wird diese HTML-Datei über das Internet an den eigenen PC übertragen und dann vom Browser interpretiert und dargestellt. Eine HTML-Datei kann man aber auch einfach auf dem eigenen PC erstellen, bearbeiten und betrachten.

\begin{Exercise}[title=Für was steht die Abkürzung HTML und was versteht man unter dem Begriff einer Auszeichnungssprache? Die Begriffe maschinenlesbare Sprache und Tag sind hier besonders wichtig., label=Auszeichnungssprache]
\end{Exercise}

Wir erstellen Schritt für Schritt unsere eigene Webseite.

\begin{Exercise}[title=Erstelle deine erste Webseite., label=DateiErstellen]

    Erstelle in deinem Heimverzeichnis (Laufwerk mit deinem Anmeldenamen) einen Ordner \textit{HTML}. Öffne den Editor und kopiere folgenden Text:

    \begin{lstlisting}
<html>

<head>

<title>Schülerseite</title>

</head>

<body>

Willkommen auf meiner Webseite!

<br>

<br>

<p>Ich bin Schüler des WG West in Stuttgart.</p>

</body>

</html>
    \end{lstlisting}
    Speichere die Datei nun als \textit{schueler.html} in deinem HTML-Ordner. (Datei\(\rightarrow\)speichern unter\(\rightarrow\)Dateityp auf Alle Dateien ändern.)
    Diese Datei kann nun mit einem Browser geöffnet werden.
\end{Exercise}

\begin{Exercise}[title=Recherchiere die Funktion der oben verwendeten Tags., label=Tags]

    Tipp: Die Webseite \href{https://www.w3schools.com/tags/default.asp}{w3schools.com} verfügt über eine Übersicht aller Tags.
\end{Exercise}

